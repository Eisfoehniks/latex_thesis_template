\thumbchapter{Einleitung}

Dies ist eine Vorlage für Abschlussarbeiten. Der Typ der Arbeit und die betreuenden Professoren sowie das Thema müssen in \path{titlepage.tex} geändert werden. Zusätzliche Kapitel und Verzeichnisse werden in \path{Master_Thesis.tex} eingefügt. Dabei sollte eine neue \path{.tex} Datei für die einzelnen Kapitel verwendet werden, da dadurch die Arbeit übersichtlicher wird.

Das Seitenlayout (Ränder, Daumenregister, Papierformat, ...) ist in \path{packages.tex} definiert und kann dort angepasst werden. Näheres dazu in den Kommentaren dieser Datei.

Der Abstract wird in \path{abstract.tex} angepasst und der Anhang befindet sich in \path{anhang.tex}.

\section{Beispiele}

Für Formeln verwendet man \lstinline|\begin{align}|.
	
\begin{align}
	e^{i \pi} &= -1 \\
	e &= \sum_{k=0}^{\infty}\frac{1}{k!}
\end{align}

Für nicht numerierte Formeln wird der Befehl \lstinline|\notag| verwendet.
\\
\autoref{diesIstEinLabel} ist ein Beispiel für ein eingebundenes Bild:

\begin{figure}[H] % H steht für "an dieser Stelle", "h" steht für "möglichst an dieser Stelle".
	\centering
	\includegraphics[width=.8\linewidth]{Bilder/mandelbrot}
	\bildquelle{Geek3, CC BY 3.0, \url{https://w.wiki/33UD}}
	\caption{Mandelbrot Rendering}
	\label{diesIstEinLabel} % Wichtig: Labels müssen immer UNTER dem Caption stehen!
\end{figure}

% Tipp für Wiki Quellen: https://meta.wikimedia.org/wiki/Special:UrlShortener


Mehrere Bilder in einer Abbildung sind auch möglich:

\begin{figure}[H]
	\centering
	\subfloat[Vogel\label{bild1}]{\includegraphics[width=0.45\linewidth]{Bilder/vogel}}%
	\qquad
	\subfloat[Kuh\label{bild2}]{\includegraphics[width=0.45\linewidth]{Bilder/kuh
	}}%
	\bildquelle{\autoref{bild1}: Andrew C, CC BY 2.0, \url{https://w.wiki/33UJ}}
	\bildquelle{\autoref{bild2}: Killarnee, CC BY-SA 4.0, \url{https://w.wiki/33UK}}
	\caption{Verschiedene Bilder in einer Grafik}%
\end{figure}

Zum Zitieren wird der \lstinline|\cite{id}| Befehl verwendet. Dabei können auch Seitenzahlen angegeben werden, z.B. \cite{Haffner2018} oder \cite[S. 15]{Haffner2018}. Der Zitierstil ist in der \path{packages.tex} Datei definiert.

\section{Nützliche Tools}

\begin{itemize}
	\item Mathematische Symbole in Latex: \newline\url{https://detexify.kirelabs.org/classify.html}
	\item Bibliographieverwaltung: z.B. \url{https://www.jabref.org/}
	\item Tabellen: \url{https://www.latex-tables.com/}
\end{itemize}

Generell ist die Verwendung von GIT für die Arbeit zu empfehlen. Durch die Verwendung von Github / Bitbucket / Gitlab lassen sich automatisch Sicherungskopien erzeugen und Versionen verwalten.