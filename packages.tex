%% Pakete laden

%% Format der Arbeit
\documentclass[enabledeprecatedfontcommands, a4paper, twoside]{scrreprt}    % articel: scrartcl   % book: scrreprt
% für einseitigen Druck (kein Buchformat) "twoside" entfernen
% für A5 Format "a5paper" verwenden
\usepackage[T1]{fontenc}
\usepackage[utf8]{inputenc}
\usepackage{fancyhdr}

%% Sprache
\usepackage[ngerman]{babel}

%% Boxstyle
\usepackage{fancybox}

%% Fußnoten
\usepackage{endnotes}

% Lesezeichenpacket für PDF
\usepackage{hyperref}

%% American Mathematical Society Pakete
\usepackage{amsmath,esint}
\usepackage{amsfonts}
\usepackage{amssymb}
\usepackage{amstext}
\usepackage{mathrsfs}
\usepackage{bbm}
\usepackage{cancel}
\DeclareMathAlphabet\mathbfcal{OMS}{cmsy}{b}{n}
\usepackage{trfsigns}

%% Grafik Paket
\usepackage{graphicx}
\usepackage{float}

%% Optional: Subfigures (mehrere Unterabbildungen in einer figure)
\usepackage{subfig}

%% Paket zum Einfügen von Quellcode (Sprache kann hier definiert werden)
\usepackage{listings}
\lstset{
	language = C++,
	showstringspaces = false,
	numbers = left,
	breaklines = true,
}

%% Zeileneinrücken verhindern
\setlength{\parindent}{0em} 

%% optional: Randlose Kapitelmarker (Daumenregister)
% Zur Verwendung für neue Kapitel \thumbchapter anstelle von \chapter verwenden
\usepackage{thumbs}
\pagenumbering{arabic}
\newcommand{\thumbchapter}[1]{
	\chapter{#1}
	\addthumb{#1}{
		\space\Huge\sffamily\bfseries \thechapter}{black}{lightgray}
}

%% optional: Quellenangabe von Bildern direkt unter der Grafik
% Zur Verwendung für Bildquellen \bildquelle verwenden
\newcommand*{\bildquelle}[1]{\par\raggedleft\footnotesize Quelle:~#1}

%% optional: Erlaubt Formatierung von Dateipfaden mit \path und URLs mit \url
\usepackage{url}

%% optional: Breiterer Rand außen
\usepackage[left=2.5cm, right=5cm, top=2.5cm, bottom=2.5cm]{geometry}

%% optional: Booktabs für schönere Tabellen
\usepackage{booktabs} % Für schöneres Tabellenaussehen

%% Style
\pagestyle{fancy}
\lhead[\bfseries \title \protect]{\bfseries \title \protect}

%% Format der Bibliographie
\usepackage{natbib}
\bibliographystyle{alpha} % Harvard Zitierweise (Autorenkürzel für Quellen)
%\bibliographystyle{abbrv} % Deutsche Zitierweise (Ziffern für Quellen)